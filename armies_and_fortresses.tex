\section{ARMIES AND FORTRESSES}

The wooden blocks represent Prussian (gray), Austrian (black), French \& Imperial (blue), and Russian/Swedish (green) forces.

Prussian labels (gray) go on the gray blocks. Austrian white labels on black blocks, French/Imperial and Saxon blue labels on blue blocks, and Russian/Swedish green labels on green blocks.

Each Fortress has two stickers per block, one with strength and one with a fortress symbol. Place one of each matching color on a block.

\subsection{Block Data}

Blocks have numbers and symbols defining movement and combat abilities.

TODO: Insert picture here.

\begin{itemize}
  \setlength\itemsep{-0.5em}
  \item Strength Points (SP) [hashes around edge]
  \item Movement Rating [lower left]
  \item Name and Class [center]
  \item Combat Power (CP) [lower right]
\end{itemize}

\subsubsection{Strength}
The number of hash marks on the top edge when the block is standing upright indicate the current strength of a block, referred hereafter as SPs. One six-sided die is rolled per army unit SP. This means a four SP unit rolls four dice.

A block’s strength is reduced by rotating the block 90 degrees counter-clockwise to lower SP. Blocks are eliminated when reduced below their printed steps and removed from play and placed in owning player’s force pool standing upright or removed from the game.

TODO: Insert picture here.

Reducing Unit Step Loss (SP)

\subsubsection{Combat Power Rating}

The Combat Power rating (CP) is shown in the lower right hand corner. The CP value is the hit value using one six sided die. A CP value of 4 means a hit is inflicted on the enemy for every 4 or higher rolled.

\textit{For example, a unit rated CP6 only scores a hit for each “6” rolled, but a unit rated CP3 scores one hit for each 3, 4, 5 or 6 rolled.}

Cavalry CP shown is for combat against cavalry. For combat against opposing leaders or infantry classes, subtract one from the CP value shown.

\textit{For example, a cavalry with a CP of 5 would score hits on 5 and 6 against cavalry and against leaders or infantry the die roll hit numbers would be 4, 5, and 6 in the charge phase.}

\subsubsection{Movement Rating}

Movement (MP) ratings are shown in the lower left corner of a block in red. This is the maximum number of cities an army unit can move.

\textit{EXCEPTIONS: Group Movement with a Leader and Force Marching.}

\subsection{Unit Types}
There are 3 types (classes) of army units: leaders, infantry, and cavalry. Fortresses are units that are not army units, but are fixed locations that have a defensive SP if assaulted.

\subsubsection{Fortresses}

Fortresses represent strong defenses, emplaced artillery, and supply depots. They do not move, are CP1 (i.e. automatically hit for each SP) and double defense (requiring two hits per SP loss).

Fortresses can be reduced in strength through assaults, siege, or if used to supply armies. Up to four army units can take refuge in a besieged fortress. Fortresses cannot move and do not count toward Winter Quartering limits. They can act as a supply source for same color blocks. Fortresses are permanent and change hands if seized. The Allied player may not later swap a captured fortress for one of a different color.

\subsubsection{Leaders}

Leaders represent the generals and their staffs, artillery, and baggage trains.

Leaders can command a number of other army units via Marshalling and Group Movement. They also can support Winter Quartering.

A leader’s MAXIMUM strength also indicates that leader’s rank relative to other leaders. Rank is important in determining which leader can move other leaders. Each leader has a Command Limit equal to TWO times their current strength. Leaders do not count against their limit.

Allied leaders may marshal other allied units as long as the number of allied army units does not exceed the number of the leader’s same nation units in the city.

\textit{Optional Rule:} Leaders with a CP of 4 or lower also may react to battles.

\subsubsection{Infantry}

Infantry are the backbone of the armies. Austrian Grenzer, light Infantry, have double defense (require two hits for each step reduction), unless assaulting or defending in a fortress.

\subsubsection{Cavalry}

Cavalry use their printed CP value against other cavalry units in the charge phase and use a CP value of six in melee against other cavalry on the BATTLE BOARD.

The Russian Cossack block is unique and can move three cities.

\subsection{Eliminated Blocks}

Eliminated army blocks due to combat or attrition, as well as eliminated fortresses, are returned to owning player’s force pool.

Eliminated leaders are permanently removed from play unless they were disbanded while in supply.

\textit{Note: If a leader is removed from play due to combining with another unit while out of supply or under siege, it is permanently removed from play.}