\bigskip
{\LARGE GAME TURN MECHANICS}

\section{CARD PHASE}
There are thirty-three cards in the game. All Cards have a Command value in left hand corner from 1 to 4. When played in the Card Phase, these are used for movement, bringing in army units from a player’s force pool, or adding steps to friendly fortresses. The higher number is for the Allies, the lower for Prussia.

All Cards have different events that may be played during movement, combat, supply, fortress siege resolution, Tsarina’s health, or Winter Phases. When played as an event, ignore the Command value. When played as Commands, ignore the event. Five of the Cards also have Replacement values printed on them where players may choose to play either as a Command card or replacement card for their steps to friendly supplied army units.

Card Mix
\begin{itemize}
  \setlength\itemsep{-0.5em}
  \item 3 - Command Prussia 3 / Allies 4
  \item 1 - Command Prussia 2 / Allies 4
  \item 4 - Command 3 for both sides
  \item 4 - Command Prussia 2 / Allies 3
  \item 15 - Command 2 for both sides
  \item 4 - Command Prussia 1 / Allies 2
  \item 2 - Command 1 for both sides
\end{itemize}

Cards are shuffled and dealt out face-down. The Campaign Game begins on Turn 3 and both players start with 4 cards.

At the beginning of each subsequent year each player gets seven cards with possible exceptions below.

If a player’s capital is occupied by the other player at the start of any year, that players receive one less card for the year.

Starting in the 1760 card phase, Prussia also receives two fewer cards if the Allied player is ahead in Victory Points on the victory point track. Once this condition is met this limit is permanent and represents the loss of British support.

Players also suffer an immediate loss of one card of their choice the moment enemy units capture their Capital.

\textit{Example: If Berlin is occupied by Austria or its allies \textbf{and} the Allied player is ahead in Victory Points at the start of 1760, then Prussia would only receive four cards for the year—a three Card reduction.}

TODO: Insert card image.

Each turn each player may select a Card to play and place it face down. Playing a card is not mandatory, but if no card is played then no Commands can be taken by that player.

The Allied player is the first to place a card face down or announce intention to pass that turn. The cards are then revealed and the player with the lowest card decides who goes first this turn. That player is Player 1.

Ties allow the Prussian player to decide who goes first. If a card is played as Replacements, that player is Player 1. If both players play Replacements, the Allied player announces first how they are playing the card – Commands or Replacements. If both players play their cards as replacements, the Prussian player is Player 1.

Neither cards nor Commands can be accumulated for future use. Once cards are played during a turn for events, they are removed from play for that year until all cards are re-shuffled dealt for a new year.

\textit{Note: Players start with extra cards per Game Year, but multiple plays of events in a year could leave a player with no cards later in the year.}

Optional Rule: For a shorter game see 11.3.

\subsection{CARDS}

\subsubsection{Replacements Cards}

There are five Replacements cards in the deck, each of which provides three to five steps of Replacements. These may be added to any friendly army units (not fortresses) located in supplied, unbesieged cities. They may be added in any combination, i.e. all steps to one army unit or steps to multiple army units in different locations. Replacements cards are always resolved first, unless played as a Command card.

\subsubsection{Movement Cards}

Six cards have events that affect movement of army units. Two cards give a +1 die roll modifier to force marching. A player could play both cards to modify a single force march attempt, gaining a +2 to the die roll.

Two cards allow an army group to ignore the restricted route limits. Both could be played for two separate groups.

\subsubsection{Combat Cards}

Thirteen of the cards have events printed on them to modify combat. These cards can be played during any field battle (not fortress assaults), but only one card per player, per battle in \textbf{any} round, except the Great Redoubt and Surprise cards that must be played in the \textbf{first} round.

Two cards allow two army groups to attack or defend in the first round. Both could be played to allow up to four groups to attack or defend together.

The attacker always announces first if they are playing a combat card then the defender. Cards are revealed simultaneously.

\subsubsection{Siege Resolution Cards}

Three of the cards can be used to affect sieges. Text on the cards explain when to play them. Players may play multiple cards to affect a single siege.

\subsubsection{Tsarina Health Cards}

Two cards allow players to modify the die roll +2 or –2 for the Tsarina’s Health roll starting in 1761. This is done before the roll and before the first turn’s card phase. The Allied player announces first if they are playing a card followed by the Prussian player. Players may play multiple cards to modify the die roll.

\subsubsection{Winter Campaign Cards}

Two of the cards allow players to choose between playing as Commands during any turn or the Winter Phase for new army units, or as the Winter Campaign event. One card is restricted to Prussia only; the other may be played by either side. The restrictions are printed on the card. The Prussian player could play both cards to move two groups.

If both players play the Winter Campaign as an event, the Prussian player decides who goes first.

\subsection{Begin New Year}
After the Winter Phase, players check for Automatic Victory. Players collect and re-shuffle all cards, then deal out the allowed 4 to 7 cards for each player.

Note key events for the year:

\textbf{1757, turn 3} – Russia, Sweden, and French/Imperial allies enter war. See setup chart.

\textbf{1760, before turn 1} – Prussia receives +1 VP if still in the war. This VP is awarded before determining if Prussia has lost British support.

\textbf{1761, turn 1} – Allied player rolls 2 dice for the death of Tsarina Elizabeth with an 8 or higher resulting in Russian and Sweden leaving the war.

\textbf{1762, before turn 1} – Prussia receives another +1 VP if still in the war.