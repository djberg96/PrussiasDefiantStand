\section{INTRODUCTION}

In 1756, Frederick the Great launched a preemptive strike on Saxony and his arch enemy Austria, igniting the Seven Years War and pitting the most powerful countries of Europe against the small fledgling state of Prussia.

Frederick’s gamble nearly destroyed Prussia, but a few stunning victories over numerically superior forces, many delaying commands, British gold, and the timely death of Tsarina Elizabeth of Russia saved Prussia and ensured Frederick’s place in history as a great general.

Prussia’s Defiant Stand recreates this fierce struggle for mastery of Central Europe. The game showcases the importance of maneuver, fortresses, sieges, and supply. The forces of Prussia, Austria, France, Russia, and Sweden are represented by blocks for Leaders / artillery, heavy and light infantry, cavalry, Cossacks, and fortresses.

One player commands the Prussian forces and the other the Austrian forces and their Holy Roman Imperial, French, and Russian allies. The object of the game is to defeat your opponent’s armies and gain control of the key cities in the war.

\subsection{GAME TURNS}

The game is played in a series of years. There are five GAME TURNS and one Winter Phase for each year except the
first year which has only three turns followed by the Winter
Phase. Each Game Turn has four PHASES as follows:

\begin{itemize}
  \setlength\itemsep{-1em}
  \item Card Phase
  \item Command Phase
  \item Battle Phase
  \item Supply Phase
\end{itemize}

\subsection{MAPBOARD}

The mapboard depicts Central Europe, including most of Prussia and Austria - Hungary, and parts of the Holy Roman Empire and Poland. The Prussian player sits at the western edge of the mapboard, the Austrian and Allies player at the eastern edge.

\subsubsection{Cities}

Cities regulate placement and movement of army units. The numeric value under some cities represents victory points and some restrictions on army unit placement. Victory Points (VPs) are awarded immediately for control of an enemy city based on printed values of 1 through 4.

The Victory Track starts at zero (0) and will slide right or left as cities change hands. All cities within the borders of Prussia and Austria are considered "Home Cities" for their respective countries. Home cities are significant for placement of new units and supply chains.

Solid lines are major routes connecting cities and dashed lines are restricted routes.

\subsubsection{City Control}

All cities are controlled by the owning country - Prussia gray, French/Imperial blue, Russia/Sweden green, and Austria white, unless the city has an enemy army unit occupying it. Unoccupied cities immediately revert to the original controlling player.

Fortresses control cities and must be taken to control a city, but only the besieging forces can trace supply through a city.

\subsubsection{Game Scale}

The map represents regions of Central Europe. Each infantry block represents a corps of 10,000 to 15,000 men, each cavalry block about 5,000 men and horse, and each leader about 100 cannon and supply wagons (25 guns per step).

\subsubsection{Fog of War}

Except when fighting a battle or conducting a siege, blocks stand upright facing the owner. Block type and strength are hidden from the opponent.

\subsubsection{Battle Sites}

The major engagements are depicted on the map with crossed rifles.